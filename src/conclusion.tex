\chapter*{Conclusion}
	\paragraph{}
	Notre problématique était de comprendre comment améliorer TrackCIS autant sur
	le plan fonctionnel (quelles améliorations apporter) que technique (comment
	apporter ces améliorations à un outil existant).
	
	\paragraph{}% conclusion de la phase d'analyse
	L'étude terrain a permis d'identifier les pratiques autour de la supervision et
	d'identifier les grands besoins des utilisateurs. Nous avons choisi de nous
	concentrer uniquement sur les besoins relatifs à un module d'affichage de
	statistiques et nous en avons déduit en ensemble de fonctionnalités. Celles-ci
	nous ont permis de modéliser le comportement du module à développer, les
	scénarios d'utilisation et les maquettes nous ont apporté une vision
	relativement détaillée des services rendues par l'outil.
	
	\paragraph{}% conclusion de la phase de conception
	Ceci nous a permis de faire des choix techniques et
	d'architectures pertinents : c'est la phase de
	conception, préalable nécessaires au développement.
	Connaissant l'architecture actuelle de
	TrackCIS, nous en avons proposé une pour le nouveau module de façon à ne pas
	modifier l'existant.
	
	\paragraph{}% conclusion de la phase de développement
	Au sein de ce projet, la production du code ne représente qu'une petite
	partie des efforts. Le développement ne commence que lorsque l'analyse et la
	conception sont terminées. Le développement en itération, inspiré des
	méthodes agiles, permet de livrer rapidement un outil fonctionnel et de
	bénéficier d'une certaine souplesse quant à la définition des fonctionnalités.
	La phase d'analyse est donc une base pour le lancement du projet mais ne
	permet pas de modéliser la version finale de l'outil. En outre, la
	conception doit pouvoir anticiper les éventuelles modifications apportées tout
	à chaque itération. C'est pourquoi l'architecture doit être la plus ouverte
	possible à l'extension :
	l'ajout d'une nouvelle fonctionnalité en cours de développement ne doit pas
	modifier l'architecture globale du module.
	
	\paragraph{}
	Ce projet s'inscrit dans une problématique plus
	vaste qui concerne l'amélioration des ventes de TrackCIS. Nous avons fait le
	choix d'aborder cette problématique sous l'angle fonctionnel uniquement. De ce
	fait nous n'avons volontairement pas traité le problème sous son aspect
	commercial. Si l'ajout de fonctionnalité ne suffit pas à déclencher l'intéret
	des utilisateurs pour TrackCIS, peut-être serait-il intéressant que la
	société Xperis réalise une étude de marché approfondie (réalisée en interne ou
	par un prestataire) sur les consoles de supervision dans le secteur de
	l'interopérabilité hospitalière de façon à avoir une meilleure compréhension
	du problème dans son ensemble.
