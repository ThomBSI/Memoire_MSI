\subsection*{Résumé}
	\paragraph{}
	Cloverleaf est un EAI (\textit{Enterprise application integration}) répondant à
	la problématique de l'interopérabilité des applications hospitalières. Cet outil,
	très complexe, nécessite des compétences spécifiques pour en assurer la
	maintenance. Les consoles de supervision telles que TrackCIS sont des outils
	plus simples à utiliser, destinés à des utilisateurs moins avertis. Les
	consoles permettent de surveiller les flux de données transitant par
	l'EAI, c'est la supervision.\\
	Ce projet a pour but d'améliorer TrackCIS en y implémentant de nouvelles
	fonctionnalités, notamment des fonctionnalités liées à l'affichage de
	statistiques sur le fonctionnement de l'EAI.
	Il se déroule en trois phases~:
	l'analyse, la conception et le développement.\\
	La phase d'analyse permet, à partir d'une enquête auprès des utilisateurs, de
	faire ressortir les principaux besoins et d'en déduire de nouvelles
	fonctionnalités. Il est ainsi possible de modéliser le comportement de
	l'outil final à l'aide de maquettes et de scénarios d'utilisation.\\
	La conception a pour but d'imaginer comment implémenter techniquement ces
	fonctionnalités. C'est lors de cette phase qu'est dessinée l'architecture de
	l'application. Celle-ci doit être ouverte au maximum à l'extension,
	c'est-à-dire permettre des améliorations futures, et doit se greffer sur
	l'architecture existante.\\
	Enfin, le développement est conduit selon la méthode Scrum, c'est-à-dire par
	itérations successives :
	les \textit{sprints}. L'objectif est d'adapter les fonctionnalités tout au long
	du développement, grâce aux retours faits par le commanditaire à la fin de
	chaque \textit{sprint}.
	
\subsection*{Abstract}
	\paragraph{}
	Cloverleaf is an EAI (Enterprise application integration), it’s a software used
	for interoperability issues in medical context. Cloverleaf is an IDE
	(Integrated development Environment), an IT dedicated tool which need training
	courses. Monitoring consoles, like TrackCIS, are more handy and user-friendly.
	They are use to monitor data flows going through the EAI.\\
	The goal of this project is to improve TrackCIS with new features, especially a
	new view with statistics. The three steps are: analysis, conception and
	development.\\
	Analysis starts with a survey about current users of monitoring consoles to
	know their needs and deduce some new features. It is possible to modeling the
	application behavior with some mock-up and use cases.\\
	Conception is about how implement these new features, design an architecture
	for the application. It must be flexible, extensible and robust. Also it must
	be consistent with existing model.\\
	Development is managed with Scrum methodology, (successive iterations or
	sprint). Final goal is to use user’s feedbacks to adapt or create features.
	
	\paragraph{Mots clés~: }
	Besoins, fonctionnalités, architecture, user stories, Scrum
	
	\paragraph{Key words~: }
	Needs, features, architecture, user stories, Scrum
	
	
	
	