\usepackage[utf8]{inputenc}
\usepackage[T1]{fontenc}
%\usepackage[francais]{babel}
\usepackage[top=2cm, bottom=2cm, right=2.5cm, left=2.5cm]{geometry}

% Figures et légendes
\usepackage{graphicx}
\usepackage{wrapfig}
\usepackage{caption}
\usepackage{float}
\floatplacement{figure}{!h}

% Police
\usepackage{fontspec}
\setmainfont{Times New Roman}

% Tableaux
\usepackage{array}
\newcommand{\comment}[1]{}
\newcommand*{\thead}[1]{%
	% Entête des tableaux
	\centering\bfseries\arraybackslash #1
}

% Page de garde
%\usepackage{fullpage}
\title{Mémoire d'ingénieur}

% Titres des parties 
\usepackage[frenchb]{babel}
\addto\captionsfrench{\renewcommand{\chaptername}{Partie}}
\renewcommand{\thechapter}{\Roman{chapter}}
\renewcommand{\thesection}{\Alph{section})}
\renewcommand{\thesubsection}{\arabic{subsection})}

% Espacements
\renewcommand{\baselinestretch}{1.15}
\setlength{\parskip}{0.1cm}
\usepackage{titlesec}
%\titlespacing\chapter{0pt}{*-300pt}{0pt}

% Maths
\usepackage{amsmath}
\usepackage{amssymb}

% Ajoute un niveau de détail de titre à la table des matières
\setcounter{tocdepth}{3}
\setcounter{secnumdepth}{3}

% Annexes
\usepackage{appendix}
\newcommand{\nocontentsline}[3]{}
\newcommand{\tocless}[2]{\bgroup\let\addcontentsline=\nocontentsline#1{#2}\egroup}

% Gestion des différentes parties du document
\makeatletter
	\newcommand\frontmatter{%
	  \cleardoublepage
	  \thispagestyle{empty}
	}
	\newcommand\mainmatter{%
	  \cleardoublepage
	  \pagenumbering{arabic}
	}
	\newcommand\backmatter{%
	  \pagenumbering{roman}
	  \if@openright
	    \cleardoublepage
	  \else
	    \clearpage
	  \fi
	}
\makeatother

% biblio
\usepackage[round,authoryear]{natbib}
\usepackage{url}