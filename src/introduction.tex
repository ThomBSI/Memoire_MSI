\section*{Introduction}
	\paragraph{}
	La gestion d'un établissement hospitalier est une tâche complexe :
	le personnel y est généralement nombreux, il faut gérer un flux continu de
	patients, gérer d'importants stocks de médicaments et autres équipements\ldots
	Le système d'information (SI) joue un rôle primordial dans la gestion d'une
	telle structure. Le SI est composé notamment des nombreux logiciels métiers
	assurant tous une ou plusieurs fonctions bien précises telles que : l'accueil
	et l'enregistrement des malades, la gestion des stocks de la pharmacie, la gestion
	des demandes et des résultats des analyses de laboratoires. Or tous ces
	outils ne sont pas nécessairement conçus pour fonctionner ensemble, ce qui
	pose un problème d'interopérabilité. Un EAI (Intégration d'Applications d'Entreprise)
	est un outil déployé à l'échelle de tout le SI et permettant de connecter
	entre elles toutes les applications métier.
	
	\paragraph{}
	Xperis est une société française spécialisée dans la distribution et le conseil
	autour de l'un de ces EAI : Cloverleaf. Cette société a également développée
	un outil annexe à Cloverleaf : TrackCIS. Cet outil permet à un utilisateur non
	averti au fonctionnement de l'EAI de néanmoins en surveiller le bon
	fonctionnement dans le temps. Or TrackCIS est un outil jeune développé il y a
	deux ans en interne et son succès commercial actuel n'est pas satisfaisant.
	C'est pourquoi Xperis cherche à l'améliorer.
	
	\paragraph{}
	Ainsi nous nous demanderons comment améliorer un outil de supervision tel que
	TrackCIS pour qu'il corresponde mieux aux besoins de ses utilisateurs ?\\
	Nous répondrons à cette problématique sous un angle à la fois fonctionnel
	(quelles sont les améliorations à apporter à l'outil ?) et technique (comment
	développer ces améliorations ?).
	
	\paragraph{}
	La première partie de ce mémoire présentera de façon détaillée la problématique
	dans son contexte. Nous y poserons les objectifs, les délimitations et les
	méthodes de notre projet.\\
	La deuxième partie traitera de l'analyse des besoins et de
	l'analyse fonctionnelle de TrackCIS. A travers ces analyses nous établirons les
	améliorations à apporter à l'outil existant.\\
	Enfin, la troisième partie sera consacrée à l'implémentation technique des
	fonctionnalités dégagées dans la partie précédente. Nous y proposerons une
	architecture ainsi qu'une méthodologie de développement, puis nous dresserons
	un bilan de ce projet et en dégagerons des pistes de travail pour la suite.
