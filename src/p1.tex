\section{TrackCIS, un outil au service de l'interopérabilité des applications hospitalières}
	\paragraph{}
	Cette première partie vise à introduire le projet sur lequel porte ce mémoire.
	Ce projet s'inscrit dans des enjeux propres à la société Xperis et porte sur
	l'outil TrackCIS, conole de supervision de Cloverleaf. Nous nous attaderons
	donc ici à présenter ces différents éléments protagonistes de notre étude de
	façon à en cerner tous les enjeux et à mettre en place une méthodologie adapté.

	\subsection{TrackCIS est une console de supervision de l'EAI cloverleaf}
		\paragraph{} 
		Commencons par définir quelques notions technique qui nous seront utiles,
		sinon indispensables pour la suite : les EAI et leurs consoles de supervision.
		 
		\subsubsection{Qu'est-ce qu'un EAI ?}
			\paragraph{}% Interopérabilité
			EAI signifi Intégration d'Application d'Entreprise. Pour bien comprendre de
			quoi il s'agit et quel rôle tient un tel outil, il faut comprendre la notion
			d'intéropérabilité. Mais avant toute chose, il est important de préciser que
			nous ne parlerons dans cette partie - et plus généralement tout au long de ce
			mémoire - que des EAI dans le monde hospitalier. de nombreuses entreprises ou
			organisme de tous les secteurs utilisent de tels outil. Dans notre cas nous nous
			focaliserons uniquement sur les EAI du monde hospitalier et nous
			chercherons à défnir les EAI au travers d'exemple de ce secteur.\newline
			Selon la définition du dictionnaire Larousse, l'interopérabilité est la "<capacité dematériels, de logiciels ou de protocoles différents à
			fonctionner ensemble et à partager des informations">. Si l'on transpose
			cette définition au monde de l'hopital, l'interopérabilité correspond à la
			capacité des différents logiciels (ou applicatifs) métier à fonctionner
			ensemble. Il existe en effet une grande diversité d'applicatifs métier dans
			ce secteur. La tableau ci-dessous offre un appercu des principaux logiciels
			que l'on peut trouver dans un hôpital.\newline
			
			\begin{tableau} %Exemples des principaux logiciels métier avec leur fonction
			% Mettre un tableau ici
			\end{tableau}
			
			Or ces différents outils ont bien souvant été développés par des sociétés
			différentes. En outre, ces outils sont spécialisés dans quelques fonctions
			biens précises : comptabilité, enregistrement des patients, gestion des
			stocks de pharmatie\ldots Dans la plupart des cas, leur conception ne leur
			permet pas de fonctionner ensemble, ce qui pourtant peut s'avérer utile. Par
			exemple, le logiciel gérant les stocks de la pharmacies peuvent avoir besoins
			d'information sur les prescriptions faites par les medecins aux patients. De
			même, les informations sur les patients saisie à l'arrivée de ces derniers
			peuvent intéresser les médecin dans le logiciel gérant les dossiers passient.
			En résumé, il est nécéssaire pour ces différentes applications de partager
			des données.
		
			\paragraph{}% L'EAI est la solution au problème de l'interopérabilité
			C'est ici qu'entre en jeux l'EAI. Il s'agit d'un outil qui établi des liens
			entre les applications métier. Il va par exemple collecter certaines données
			émises par le logiciel de gestion des entrées patients pour les injecter dans
			l'outil de gestion des dossiers patients. Plus généralement, il transfert des données
			émisent par un outil A pour les mettre à disposition d'un outil B.\newline
			Mais les fonctions de l'EAI ne se résume pas au seul transport des
			données. Il est parfois utile de réaliser des actions sur ls informations
			transmises d'une application à l'autre. Ces actions peuvent êtres par exemple
			:\newline
			\begin{itemize}
			  \item de valider les informations transmises. Par exemple, l'EAI va
			  s'assurer que dans chaque donnée émise par une application de gestion des
			  prescription, le nom du patient est bien présent.
			  \item d'effectuer des transformation. L'application de destination de
			  gère pas forcément les mêmes frmats de donnée que l'application émétrice
			  car, rappelons le, les applications ne sont pas forcéments conçues pour
			  fonctionner ensemble. L'EAI est ainsi capable de faire des conversion de
			  format de donnée.
			\end{itemize}

			\paragraph{}% Un peu de vocabulaire
 			Arrêtons nous un instant pour préciser le vocabulaire que nous utiliserons
 			par la suite.\newline
 			\begin{description}
 				\item[Flux] Les flux sont tout simlement les routes permettant de connecter
 				une application à une autre. Dans un système d'information hospitalier
 				(SIH), il y a donc autant de flux de de couple d'application. Il est
 				possible - et même courent - qu'une application soit reliée à plusieurs
 				autres.
 				\item[Message] Les données qui passent le long des flux, d'une application
 				à une autre, sont appelés message. Les messages se présente en général sous
 				la forme de text hautement standardisé. les messages sont émis par les
 				applicatifs métier dans un format donné. Il existe de nombreux formats
 				dont certains sont spécifiques au monde hospitalier. L'annexe N présente
 				quelques uns des principaux formats utilisés (HL7 et HPRIM).
 			\end{description}
 			
		\subsubsection{Les flux de messages doivent être surveillés}
			\paragraph{}
			
			
		\subsubsection{TrackCIS est un outil qui permet la supervision des flux}
			\paragraph{}
			Texte de la sous sous partie
		
		% Résumé de la partie 1.1
		Nous avons vu dans cette première partie que l'EAI est une solution au
		problème de l'onteropérabilité en établissant des flux de données entre
		applications métier. Nous avons souligné le fait que la surveillance (la
		supervision) continue des flux est indispensable au bon fonctionnement de
		l'EAI et, par extention, de tout le SI. Enfin, TrackCIS est un outil qui permet de rendre la
		supervision plus simple et plus rapide.\newline
		Dans la suite nous découvrirons les prblématiques inhérentes à cet outil.
		
	\subsection{TrackCIS est au cœur d'une problématique commerciale pour Xperis}
		\paragraph{}
		Précédement nous avons établi les bases des problématiques d'interopérabilité
		des établissements hospitalier. Xperis est une entreprise qui propose
		d'accompagner ces dernier dans la mise en place et le maintien d'un EAI :
		Cloverleaf.
		
		\subsubsection{Xperis est l'intermédiaire dans la distribution de Cloverleaf}
			\paragraph{}% Présentation de Cloverleaf
			Xperis est une entreprise Bordelaise spécialisée dans le
			consulting autour d'un EAI du nom de Cloverleaf.
			
			\paragraph{}% Position d'Xperis sur le marché
			
			\paragraph{}% Rapide historique d'Xperis
			
			
		\subsubsection{Xperis cherche à établir des liens directs avec les hôpitaux}
			\paragraph{}% panorama du monde hospitalier
			
			\paragraph{}% regroupement des hôpitaux en GHT
			
			\paragraph{}% Maincaire va peut-être se séparer d'Xperis un jour
			
			\paragraph{}% La stratégie d'Xperis face à ces challenges
			
		\subsubsection{Un outil qui se vend mal et qui est peu utilisé}
			\paragraph{}% 
			
			% Résumé de la partie 1.2
			\paragraph{}
			Nous avons vu qu'Xperis est confronté à un certain nombre de problématiques
			commerciales. L'entreprise à une vrais volonté de traiter en direct avec les
			hôpitaux pour gagner en indpendance vis à vis des éditeurs. TrackCIS entre
			dans cette stratégie mais a du mal à se faire une place parmi les
			clients.\newline
			L'ensemble de ces problématiques constituent le socle du présent travail.
	
	\subsection{Vers une nouvelle version de TrackCIS}
		\paragraph{}% problème => solution commerciale ou solution fonctionnelle
		Le poblème que pose TrackCIS soulève une question : pourquoi cet outil ne se
		vend t-il pas ? La réponse à cette question, loin d'être triviale, pourrait
		être de deux type : une solution commerciale et une solution fonctionnelle.
		\begin{description}
			\item[La solution commerciale] Nous faisons ici l'hypothèse que la cause au
			fait que TrackCIS ne se vende pas est commerciale. Ceci pouvant impliquer
			plusieurs chose comme le fait que :
			\begin{itemize}
			  \item les clients n'ont pas réellemen besoins d'un outil tel que TrackCIS.
			  Il esxiste en effet d'autres outils de supervision pour Cloverleaf, comme
			  par exemple l'outil Global Monitore, dont nous reparlerons un peu plus
			  loin, ou encore EAI Supervision.
			  \item les utilisateur de TrackCIS n'existent pas. Cet outil a été développé
			  pour des personnes non initiées à Cloverleaf mais qui pour qui il serait
			  intéressant de superviser quelques flux.
			  Or il est tout à fait possible que ce type de personne n'existe pas ou pas
			  encore dans les hôpitaux.
			  \item la communication ou le démarchage autour de TrackCIS est insufisant
			  ou non adapté.
			\end{itemize}
			\item[La solution fonctionnelle] Cette fois-ci notre hypothèse est que
			TrackCIS ne se vend pas car il lui manque des fonctionnalité indispensables à
			ses utilisateurs.
		\end{description}
		
		\subsubsection{Comprendre les utilisateurs et leurs besoins}
			\paragraph{}
			Texte de la sous sous partie
		\subsubsection{Un module statistique lié aux évolutions de Cloverleaf}
			\paragraph{}
			Texte de la sous sous partie
		\subsubsection{Méthodologie générale du projet}
			\paragraph{}
			Texte de la sous sous partie
			