\section{TrackCIS, un outil au service de l'interopérabilité des applications hospitalières}
	\paragraph{}
	Cette première partie vise à introduire le projet sur lequel porte ce mémoire.
	Ce projet s'inscrit dans des enjeux propres à la société Xperis et porte sur
	l'outil TrackCIS, conole de supervision de Cloverleaf. Nous nous attaderons
	donc ici à présenter ces différents éléments protagonistes de notre étude de
	façon à en cerner tous les enjeux et à mettre en place une méthodologie adapté.

	\subsection{TrackCIS est une console de supervision de l'EAI cloverleaf}
		\paragraph{}
		Commencons par définir quelques notions technique qui nous seront utiles,
		sinon indispensables pour la suite : les EAI et leurs consoles de supervision.
		
		\subsubsection{Qu'est-ce qu'un EAI ?}
			\paragraph{}% Interopérabilité
			EAI signifi Intégration d'Application d'Entreprise. Pour bien comprendre de
			quoi il s'agit et quel rôle tient un tel outil, il faut comprendre la notion
			d'intéropérabilité. Mais avant toute chose, il est important de préciser que
			nous ne parlerons dans cette partie - et plus généralement tout au long de ce
			mémoire - que des EAI dans le monde hospitalier. de nombreuses entreprises ou
			organisme de tous les secteurs utilisent de tels outil. Dans notre cas nous nous
			focaliserons uniquement sur les EAI du monde hospitalier et nous
			chercherons à défnir les EAI au travers d'exemple de ce secteur.\newline
			Selon la définition du dictionnaire Larousse, l'interopérabilité est la "<capacité dematériels, de logiciels ou de protocoles différents à
			fonctionner ensemble et à partager des informations">. Si l'on transpose
			cette définition au monde de l'hopital, l'interopérabilité correspond à la
			capacité des différents logiciels (ou applicatifs) métier à fonctionner
			ensemble. Il existe en effet une grande diversité d'applicatifs métier dans
			ce secteur. La tableau ci-dessous offre un appercu des principaux logiciels
			que l'on peut trouver dans un hôpital.\newline
			
			Or ces différents outils ont bien souvant été développés par des sociétés
			différentes. En outre, ces outils sont spécialisés dans quelques fonctions
			biens précises : comptabilité, enregistrement des patients, gestion des
			stocks de pharmatie\ldots Dans la plupart des cas, leur conception ne leur
			permet pas de fonctionner ensemble, ce qui pourtant peut s'avérer utile. Par
			exemple, le logiciel gérant les stocks de la pharmacies peuvent avoir besoins
			d'information sur les prescriptions faites par les medecins aux patients. De
			même, les informations sur les patients saisie à l'arrivée de ces derniers
			peuvent intéresser les médecin dans le logiciel gérant les dossiers passient.
			En résumé, il est nécéssaire pour ces différentes applications de partager
			des données.
		
			\paragraph{}% L'EAI est la solution au problème de l'interopérabilité
			C'est ici qu'entre en jeux l'EAI. L'EAI est un outil qui établi des liens
			entre les applications métier. Il va par exemple collecter certaines données
			émises par le logiciel de gestion des arrivés pour les injecter dans l'outil
			de gestion des dossiers patients. Plus généralement, il transfert des données
			émisent par un outil A pour les mettre à disposition d'un outil B.\newline
			Mais l'EAI ne se résume pas seulement au transport des données.

			\paragraph{}% Un pe de vocabulaire
 			Arrêtons nous un instant pour préciser le vocabulaire que nous utiliserons
 			par la suite.\newline
 			\begin{description}
 				\item[Flux]
 				\item[Message]
 				\item[Connecteur]
 			\end{description}
 			
		\subsubsection{Les flux de messages ont besoin d'être surveillés}
			\paragraph{}
			Les logiciels de santé sont de nature très diverses. Au sain d'un
 			établissement hospitalier, l'on trouve par exemple :
 			\begin{itemize}
 			  \item{-} Des outils dit 
 			\end{itemize}
		\subsubsection{TrackCIS est un outil qui permet la supervision des flux}
			\paragraph{}
			Texte de la sous sous partie
	
	\subsection{TrackCIS est au cœur d'une problématique commerciale pour Xperis}
		\paragraph{}
		Introduction de la sous partie.
		
		\subsubsection{Xperis est l'intermédiaire dans la distribution de Cloverleaf}
			\paragraph{}
			Texte de la sous sous partie
		\subsubsection{Xperis cherche à établir des liens directs avec les hôpitaux}
			\paragraph{}
			Texte de la sous sous partie
		\subsubsection{Un outil qui se vend mal et qui est peu utilisé}
			\paragraph{}
			Texte de la sous sous partie
	
	\subsection{Vers une nouvelle version de TrackCIS}
		\paragraph{}
		Introduction de la sous partie.
		
		\subsubsection{Comprendre les utilisateurs et leurs besoins}
			\paragraph{}
			Texte de la sous sous partie
		\subsubsection{Un module statistique lié aux évolutions de Cloverleaf}
			\paragraph{}
			Texte de la sous sous partie
		\subsubsection{Méthodologie générale du projet}
			\paragraph{}
			Texte de la sous sous partie
			