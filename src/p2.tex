\newpage
\section{Phase d'analyse : des besoins aux fonctionnalités}
	\paragraph{}
	Intorduction de la partie.
	
	\subsection{L'enquête permet une meilleure compréhension des besoins des utilisateurs}
		\paragraph{}
		Introduction de la sous partie.
		
		\subsubsection{Élaboration d'un questionnaire d'enquête et conduite des entretiens}
			\paragraph{}% Les objectifs de l'enquêtes
			L'enquête poursuit deux objectifs :
			\begin{itemize}
			  \item D’une part nous souhaitons pouvoir établir une liste de besoins.
			  \item D’autre part, cette enquête sera l’occasion de comprendre un peu
			  mieux comment sont organisés les DSIO (direction des services informatique
			  et de l’organisation) des établissements hospitaliers, ainsi que les types
			  d’utilisateurs des consoles de supervision.
			\end{itemize}
			
			\paragraph{}% Les grands thèmes abordés
			Pour répondre à ces deux objectif, le questionnaire d’enquête doit être le
			plus générale possible. Ainsi, nous conduirons les entretiens en suivant le
			plan de questionnaire suivant :
			\begin{itemize}
			  \item[- Identification de la personne, exploration du contexte] . Cette
			  partie aura pour but de nous aider à comprendre comment est organisé le
			  service et qui est notre interlocuteur.
			  \item[- Utilisations et utilisateurs de Cloverleaf et problématique de la
			  supervision des flux] . Dans cette partie nous poserons des questions sur
			  la manière dont est utilisé Cloverleaf et par qui ainsi que les personnes
			  qui sont en charge de la supervision.
			  \item[- Attentes par rapport aux consoles de supervision] Cette partie,
			  beaucoup plus abstraite, servira à comprendre ce qu’est la supervision
			  pour les utilisateurs de consoles et quels sont les attentes des
			  utilisateurs sur les consoles de supervision.
			  \item[- Les consoles actuelles] Ces questions viseront à explorer ce que
			  les utilisateurs utilisent le plus dans les outils existant, quels sont
			  les manques et les améliorations qu’ils proposent. Cette partie explorera
			  également les propositions des utilisateurs concernant un future module
			  statistique.
			\end{itemize}
			
			\paragraph{}% Conduite des entretiens
			Les entretiens sont conduit de manière semi-directive par téléphone ou en
			face à face quand cela est possible.
			Les questions ne sont que des guides pour orienter la discussion, c’est
			pourquoi elles sont les plus ouvertes possible.\newline
			Le questionnaire a vocation à évoluer d’un entretien à un autre : des
			questions peuvent être modifiées, supprimées ou ajoutées de façon à obtenir
			des informations de plus en plus pertinentes. Chaque entretien donne lieu à
			un compte-rendu détaillé dont la structure reprend la trame du
			questionnaire. Le questionnaire, dans sa version finale, se trouve en annexe
			A.
			
			\paragraph{}% Les principes de l'étude qualitative
			Il s'agit d'une étude qualitative. Contrairement aux études quantitatives qui
			permettent de récolter des données repérsentatives d'une population,
			dans une étude qualitative nous ne cherchons pas à interroger le plus de
			monde possible. Les entretients sont basés sur des question ouverte et durent
			en général plus longtemp que dans des enquêtes quantitatives. L'objetcif
			d'une telle étude est de faire ressortir la diversité des comportement. Dans
			notre cas, il s'agit de mettre en avant les différentes utilisations
			possibles des consoles de supervision et ses utilisateurs.
			
			\paragraph{}% Choix de la population interrogée
			Un panel de 5 établissements hospitaliers est interrogé. Tous ces
			établissements sont des clients d’Xperis :
			\begin{itemize}
			  \item Tour
			  \item Toulouse
			  \item Rouen
			  \item Brest
			  \item Metz
			\end{itemize}
			
			\paragraph{}% Méthodes d'analyse des résultat
			A l'issue des enquêtes nous procéderons à l'analyse des résultats. Après une
			rapide relecture de tous les comptes rendu d'entretien, nous établissons une
			listes des grands thèmes qui ont été abordés durant les discussion. Nous
			classons ensuites toutes les informations collectées dans ces thèmes.\newline
			Les thèmes qui sont ressortis sont les suivants :
			\begin{itemize}
			  \item L’organisation des services informatique dans les hôpitaux,
			  \item Les types d’utilisateurs de TC,
			  \item Les types d’utilisations de TC,
			  \item Les attentes par rapport aux consoles de supervision et la réponse
			  des outils actuels à ces attentes,
			  \item Les fonctionnalités manquant aux consoles de supervisions
			  actuelles,
			  \item Les statistiques sur les flux.
			\end{itemize}
			Dans la suite de cette partie nous entrerons dans le détail des informations
			collectées pour chacun de ces grand thèmes.
			
		\subsubsection{La supervision fait partie du processus de résolution des
		anomalies}
			\paragraph{}% Présentation
			
			\paragraph{}% L'organisation des DSIO
			
			\paragraph{}% Diversité des pratiques
			
		\subsubsection{Une grande diversité des utilisateurs des consoles de
		supervision}
			\paragraph{}% Présentation
			
			\paragraph{}% Les utilisateurs
			
		\subsubsection{L'enquête permet de dresser une liste des besoins}
			\paragraph{}% La notion de besoin
			
			\paragraph{}% Les attentes qui sont ressorties de l'enquête
			
			\paragraph{}% La lise des besoins
			A partir des entretiens, nous avons dresser une liste de besoins de natures
			diverse. Ceux-ci consernent différents aspect de TrackCIS ou de la
			supervision. On peut les classer dans différentes catégories :
			\begin{itemize}
			  \item Les besoins
			\end{itemize}
	
	\subsection{La liste des besoins permet de dresser une liste de fonctionnalités}
		\paragraph{}% On ne se concentre que sur les besoins liés aux stats
		Nous avons dressé et analysé une liste de besoins émanant d'utilisateur
		concernant la supervision. Notre projet ne vise pas à traiter l'intégralité de
		ces besoins. Nous ne nous focaliserons ici que sur les besoins relatifs à un
		module de consultation de statistiques.
		
		\subsubsection{De la liste des besoins à la liste des fonctionnalités}
			\paragraph{}% Mode de passage des besoins aux fonctionnalités
			
			\paragraph{}% Formalisation des fonctionnalités grace au diagramme FODA
			
		\subsubsection{Priorisation des fonctionnalités liées aux statistiques}
			\paragraph{}% But de la priorisation des fonctionnalités
			
			\paragraph{}% Quelques rappels en UML
			
			\paragraph{}% Représentation des fonctionnalités liées aux stats en UML
	
	\subsection{Des fonctionnalités vers les cas d'utilisation}
		\paragraph{}
		Introduction de la sous partie.
		
		\subsubsection{État de l'art sur les consoles de supervision et la visualisation de données}
			\paragraph{}
			Texte de la sous sous partie
		\subsubsection{Choix des modes de visualisation par le biais de maquettes}
			\paragraph{}
			Texte de la sous sous partie
		\subsubsection{Création de scénarios d'utilisation}
			\paragraph{}
			Texte de la sous sous partie
