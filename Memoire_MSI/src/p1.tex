\section{TrackCIS, un outil au service de l'interopérabilité des applications hospitalières}
	\paragraph{}
	

	\subsection{TrackCIS est une console de supervision de l'EAI cloverleaf}
		\paragraph{}
		Introduction de la sous partie.
		
		\subsubsection{Qu'est-ce qu'un EAI ?}
			\paragraph{} % Définition de l'EAI
			EAI signifi intégration d'application d'entreprise. Il s'agit d'un outil
			permetant à des applications très différentes, développées par des éditeurs
			différents, de communicer entre elles. L'EAI peut être vu comme un réseau
			routier entre les applications d'un système d'information. Chacune de ces
			route transportant des données, autrement appelés messages, d'un outil à un
			autre. Pour établir ces routes, l'EAI doit pouvoir se brancher sur les
			différens outils.
			Chacune des route à ainsi pour vocation de transporter des données, mais la
			fonction de l'EAI ne se limite pas au transport. Il est en effet possible de
			faire des transformations sur les données durant leur transfert d'un
			application A vers une application B. Il peut s'agir par exemple de modifier
			le format des données émises par l'outil A pour les adapter au format de
			donnée géré par l'outil B.
			
			\paragraph{} % A quoi servent les EAI et où peut-on les trouver ?
			Notre travail porte sur l'EAI Cloverleaf qui est spécifique au secteur
			hospitalier. Mais les EAI ne se résume pas au monde de la santé, l'on peut en
			trouver dans d'autres secteurs, et plus généralement dans les grandes
			entreprises ou organisation.
			L'EAI permet d'intégrer des applicatifs fonctionnant de façon autonome dans
			un système d'information globale. C'est l'intéropérabilité.
			Le présent travail ne porte que sur l'EAI Cloverleaf et son application pour
			le monde de la santé.
			
			
			
			
		\subsubsection{Les flux de messages ont besoin d'être surveillés}
			\paragraph{}
			Les logiciels de santé sont de nature très diverses. Au sain d'un
			établissement hospitalier, l'on trouve par exemple :
			\begin{itemize}
			  \item{-} Des outils dit 
			\end{itemize}
			
			
			
			
			
		\subsubsection{TrackCIS est un outil qui permet la supervision des flux}
			\paragraph{}
			Texte de la sous sous partie
	
	\subsection{TrackCIS est au cœur d'une problématique commerciale pour Xperis}
		\paragraph{}
		Introduction de la sous partie.
		
		\subsubsection{Xperis est l'intermédiaire dans la distribution de Cloverleaf}
			\paragraph{}
			Test
			
		\subsubsection{Xperis cherche à établir des liens directs avec les hôpitaux}
			\paragraph{}
			Texte de la sous sous partie
		\subsubsection{Un outil qui se vend mal et qui est peu utilisé}
			\paragraph{}
			Texte de la sous sous partie
	
	\subsection{Vers une nouvelle version de TrackCIS}
		\paragraph{}
		Introduction de la sous partie.
		
		\subsubsection{Comprendre les utilisateurs et leurs besoins}
			\paragraph{}
			Texte de la sous sous partie
		\subsubsection{Un module statistiques lié aux évolutions de Cloverleaf}
			\paragraph{}
			Texte de la sous sous partie
		\subsubsection{Méthodologie générale du projet}
			\paragraph{}
			Texte de la sous sous partie
			