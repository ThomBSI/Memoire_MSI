\section{Phase d'analyse : des besoins aux fonctionnalités}
	\paragraph{}
	Intorduction de la partie.
	
	\subsection{L'enquête permet une meilleure compréhension des besoins des utilisateurs}
		\paragraph{}
		Introduction de la sous partie.
		
		\subsubsection{Élaboration d'un questionnaire d'enquête et conduite des entretiens}
			\paragraph{}
			Texte de la sous sous partie
		\subsubsection{Une grande diversité des utilisateurs des consoles de supervision}
			\paragraph{}
			Texte de la sous sous partie
		\subsubsection{La supervision fait partie du processus de résolution des anomalies}
			\paragraph{}
			Texte de la sous sous partie
		\subsubsection{L'enquête permet de dresser une liste des besoins}
			\paragraph{}
			Texte de la sous sous partie
	
	\subsection{La liste des besoins permet de dresser une liste de fonctionnalités}
		\paragraph{}
		Introduction de la sous partie.
		
		\subsubsection{De la liste des besoins à la liste des fonctionnalités}
			\paragraph{}
			Texte de la sous sous partie
		\subsubsection{Priorisation des fonctionnalités liées aux statistiques}
			\paragraph{}
			Texte de la sous sous partie
	
	\subsection{Des fonctionnalités vers les scénarios d'utilisation}
		\paragraph{}
		Introduction de la sous partie.
		
		\subsubsection{État de l'art sur les consoles de supervision et la visualisation de données}
			\paragraph{}
			Texte de la sous sous partie
		\subsubsection{Choix des modes de visualisation par le biais de maquettes}
			\paragraph{}
			Texte de la sous sous partie
		\subsubsection{Création de scénarios d'utilisation}
			\paragraph{}
			Texte de la sous sous partie
